% ----------------------------------------------------------------
% achemso --- Support for submissions to American Chemical
%  Society journals
% Maintained by Joseph Wright
% E-mail: joseph.wright@morningstar2.co.uk
% Originally developed by Mats Dahlgren
%  (c) 1996-98 by Mats Dahlgren
%  (c) 2007-2008 Joseph Wright
% Released under the LaTeX Project Public license v1.3c or later
% See http://www.latex-project.org/lppl.txt
% 
% Part of this bundle is derived from cite.sty, to which the
% following license applies:
%   Copyright (C) 1989-2003 by Donald Arseneau
%   These macros may be freely transmitted, reproduced, or
%   modified provided that this notice is left intact.
% ----------------------------------------------------------------
% 
% The achemso bundle provides a LaTeX class file and BibTeX style
% file in accordance with the requirements of the American
% Chemical Society.  The files can be used for any documents, but
% have been carefully designed and tested to be suitable for
% submission to ACS journals.
% 
% The bundle also includes the natmove package.  This package is
% loaded by achemso, and provides automatic moving of superscript
% citations after punctuation.

\documentclass[
%journal=ancac3, % for ACS Nano
%journal=acbcct, % for ACS Chem. Biol.
journal=jacsat, % for undefined journal
manuscript=article]{achemso}

\usepackage[version=3]{mhchem} % Formula subscripts using \ce{}

\newcommand*{\mycommand}[1]{\texttt{\emph{#1}}}

\author{George Mathew}
\email{george2@ncsu.edu}
\affiliation[North Carolina State University]
{Department of Computer Science, North Carolina State University}


\title[\texttt{Py*} abstract]
{Py* : Python Based Modelling Module For Requirements Engineering}

\begin{document}

\begin{abstract}
In early phase of software development, mapping functional requirements are critical since this
gives a high level insight on the feasibility of the software. UML diagrams were used initially
for modelling, but UML often focuses on organisational objects, which are not
so important in the early phase, when the emphasis should be on helping stakeholders gain better
understanding of the various possibilities for using information systems in their organizations.
The i* model was proposed in 2005 to provide an early understanding of the organizational
relationships in a business domain. The Use Case development from organizational modeling using
i* allows requirement engineers to establish a relationship between the functional requirements
of the intended system and the organizational goals. Constructing these models using the current state of the art (OpenOME developed in 2008) is cumbersome, time consuming and makes it hard to perform predictive analysis and optimization on the models. In this project a simple python based module is proposed which can be easily used to create and update the models by the business user based on his requirements. At the same time \textbf{Py*} also reduces the effort of the requirements engineer to predict the outcome of the models and optimize for its feasibility.
\end{abstract}

\end{document}
